% Template version as of 6/27/2024
\documentclass[conference, 12pt]{IEEEtran} 
\IEEEoverridecommandlockouts
% The preceding line is only needed to identify funding in the first footnote. If that is unneeded, please comment it out.
\usepackage{cite}
\usepackage{amsmath,amssymb,amsfonts}
\usepackage{algpseudocode}
\usepackage{algorithm}
\usepackage{graphicx}
\usepackage{textcomp}
\usepackage{booktabs}
\usepackage{listings}
\usepackage{csquotes}
\usepackage{hyperref}
\usepackage{xcolor}
\usepackage{fontawesome}
\usepackage{cleveref}

\crefformat{figure}{Fig.~#2#1#3}

\begin{document}

\title{
Simulating Market Microstructure and Algorithmic Behavior in a Continuous Double Auction
}

\author{
\IEEEauthorblockN{
Anonymous Submission
}

% \IEEEauthorblockN{
% % 1\textsuperscript{st}
% Arush Tagade
% % \textsuperscript{\href{mailto:xxx@example.com}
% % {\faEnvelopeO}
% % }\
% }

% \and

% \IEEEauthorblockN{
% % 2\textsuperscript{nd}
% Aizierjiang Aiersilan
% }

% \and

% \IEEEauthorblockN{
% % 3\textsuperscript{nd}
% Swapnaneel Chatterjee
% }

}



\maketitle



\begin{abstract}
We present a continuous double-auction (CDA) exchange simulator implemented in the Go programming language, designed to investigate emergent market phenomena generated by heterogeneous populations of autonomous, rule-based agents. These agents are deliberately modeled as low- or zero-intelligence, following simple rules for submitting and canceling orders. The simulator enables variation in agent distribution and behavioral parameters, including inventory constraints, trend-following strategies, and execution rules, to examine how agent composition shapes market outcomes. By running controlled experiments, we highlight how algorithmic behaviors and microstructure rules can jointly produce instabilities, including flash-crash-like dips. The framework further provides a platform for evaluating potential policy interventions—such as price bands or limits on trading frequency—as mechanisms for mitigating instability. Leveraging Go’s concurrency model, the implementation supports high-throughput, massively parallel simulations, offering a flexible and computationally efficient tool for empirical research into market design and regulatory mechanisms. This initial version outlines the design principles, scope, and extensions of the project, with further work to refine experimental methodology, evaluate performance, and analyze policy implications.
\end{abstract}

\section{Introduction}
Financial markets are increasingly understood as complex adaptive systems where microstructure—the specific rules, institutions, and procedures of trade execution—plays a decisive role in shaping outcomes, often alongside or even independent of macroeconomic fundamentals \cite{hasbrouck2007empirical}. The rapid rise of algorithmic and high-frequency trading has amplified this complexity: autonomous programs now constitute the majority of orders in many modern exchanges \cite{qian2007quantitative}. While algorithmic trading has improved liquidity and efficiency, it has also exposed markets to fragility, as evidenced by the 2010 “Flash Crash” in the United States, during which the Dow Jones Industrial Average plunged nearly 1000 points within minutes before rebounding \cite{kirilenko2017flash}. Recent agent-based high-frequency simulation studies replicate such events and analyze their dynamics with millisecond granularity \cite{gao2022high}.

\section{Background and Related Work}
\subsection{Zero-Intelligence Traders and Market Efficiency}
The foundational contribution of Gode and Sunder \cite{gode1993allocative} demonstrated that allocative efficiency close to 100\% can be achieved even when agents place bids and asks randomly, provided they respect budget constraints. This finding challenged the assumption of rational expectations and showed that the continuous double auction mechanism itself generates efficient allocations. More recent empirical game-theoretic analyses test the stability of non-adaptive (ZI-like) strategies in CDA environments \cite{wright2018evaluating}.

\subsection{Heterogeneous Trader Models}
Subsequent studies emphasized heterogeneity in agent rules. Chiarella and Iori \cite{chiarella2009impact} formalized heterogeneous agents (e.g., fundamentalists, chartists, noise traders) to study price dynamics, volatility, and liquidity under continuous double auction systems. These works highlight the sensitivity of emergent phenomena to the distribution of trading rules. More recent intra-day agent-based frameworks reproduce stylized facts across time-scales \cite{staccioli2021agent}.

\subsection{Agent-Based Flash Crash Simulations}
High-frequency agent-based simulators provide a compelling platform to study flash crashes. Gao et al. (2024) develop a millisecond-scale simulator calibrated for the E-Mini S\&P 500 futures market, successfully reproducing both the 2010 Flash Crash and “mini” flash crashes, and isolating key drivers such as sell-algorithm volume, market-maker inventory limits, and fundamental trader frequency \cite{gao2022high}. Additionally, hybrid micro-macro models investigate flash-crash contagion and systemic risk in multi-asset scenarios \cite{paulin2019understanding}. Related studies highlight adversarial feedback loops and systemic fragilities arising from portfolio crowding and network topology \cite{paulin2019understanding}.

\subsection{Learning and Adaptive Agents in CDA}
Recent approaches incorporate learning mechanisms into agent-based CDA environments. For example, Mahfouz et al. (2021) use behavioural cloning to classify and imitate trading agent archetypes within limit order book simulations \cite{mahfouz2021learning}. Other work employs reinforcement learning to train market-making agents in LOB simulations, balancing inventory risk versus profitability \cite{beysolow2019market}. These methods enrich the agent types beyond static heuristic behaviors.

\subsection{ABM for Regulatory and Risk Scenarios}
Agent-based simulations have gained traction in regulatory discussions and systemic risk assessment. [TBD](Some related works to be mentioned for this scope).

\section{Scope}
\subsection{Minimum Viable Product (MVP)}
The initial implementation includes:
\begin{itemize}
  \item \textbf{Core exchange system}: A single goroutine manages the mutable order book state, ensuring strict FIFO event processing.
  \item \textbf{Order book structure}: Two per-stock main queues (best bid and best ask), with per-price FIFO queues for resting orders.
  \item \textbf{Order types}:
    \begin{itemize}
      \item \textbf{Limit (GTC) orders}: Execute at the specified price or better; remain until filled or canceled.
      \item \textbf{Market (IOC) orders}: Execute immediately at the best available price; cancel any unfilled portion.
    \end{itemize}
  \item \textbf{Agent operations}: Submit order and cancel order.
  \item \textbf{Agent types}:
    \begin{itemize}
      \item \textbf{Market Makers}: Full knowledge, providing liquidity on both sides.
      \item \textbf{Trend Followers}: Partial knowledge, reacting to price trends.
      \item \textbf{Random Agents}: No knowledge, analogous to ZI traders.
    \end{itemize}
  \item \textbf{Validations}: Ensure non-negative quantities and prices.
\end{itemize}

\subsection{Extensions}
Planned extensions include:
\begin{itemize}
  \item Incorporating real historical stock price data for calibration and validation [TBD] (e.g. agent-based calibration techniques).
  \item Implementing transaction security features [TBD] (e.g. secure matching protocol frameworks).
  \item Expanding policy analysis to include mechanisms such as circuit breakers, order throttling, and batch auctions [TBD] (e.g. inspired by ABM regulatory studies).
  \item Introducing adaptive agents via reinforcement learning or imitation techniques \cite{mahfouz2021learning, beysolow2019market}.
\end{itemize}

\subsection{Performance Targets}
Performance benchmarks are currently [TBD] (e.g. aim for sub-millisecond event processing and parallel agent scalability, comparable to Gao et al.'s simulation design).

% \section{Methodology}

% \section{Evaluation}


\section{Conclusion}
We introduce a continuous double auction simulator in Go for studying the interplay between agent heterogeneity, market microstructure, and systemic stability. By systematically varying the composition and behavioral rules of rule-based agents, the framework enables controlled exploration of emergent instabilities and evaluation of policy levers. Future work will extend the simulator to incorporate historical data, security features, machine learning-driven agents, and detailed performance benchmarking.



%%%%%%%%% REFERENCES %%%%%%%%%
\newpage
{\small
\bibliographystyle{IEEETran}
\bibliography{IEEEfull}
}
\end{document}